\documentclass[UTF8]{ctexart}
\usepackage{dirtree}
\usepackage{listings}
\linespread{1}%修改行距
\title{编译原理实验一:词法分析与语法分析} 
\author{1190200708 熊峰} 
\date{\today}
\begin{document} 
\maketitle 

\section{程序的功能及实现} 
\subsection{程序的功能}
使用flex和bison实现C--语言的词法和语法分析,构造分析树并能够在发现词法错误和语法错误时给出提示。\par
\subsection{程序的实现}

所设置的节点的结构体如下:\par
\lstset{language=C}
    \begin{lstlisting}
        typedef struct treeNode{
            // 词法单位的行
            int lineNo;
            // 词法单位的类型
            NodeType nodeType;
            // 词法单位的名称
            char* name;
            // 词法单位若为终结符则保存他的内容
            char* value;
            // 非终结符的孩子节点
            struct treeNode* childNode;
            // 非终结符的下一个兄弟节点
            struct treeNode* brotherNode;
        }Node;
    \end{lstlisting} 
\subsubsection{节点相关操作}
    \begin{itemize}
        \item [1)]
        创建节点,并根据value的值区分是否为终结符。\par
    
        \item [2)]
        插入节点,首先检查父节点是否存在子节点,若不存在子节点,则将第二个参数的节点插入为父节点的子节点,若其存在
        子节点,则找到父节点的子节点,并不断指向兄弟节点,直到其为空,将第二个参数的节点插入为最后一个兄弟节点的兄弟节点。\par
    
        \item [3)]
        删除语法树,释放内存。\par
        \item [4)]
        根据层数打印语法树,确定节点类型,只有为TOKEN\_INT,TOKEN\_\par
        FLOAT,TOKEN\_ID,TOKEN\_TYPE,的时候才会打印他的值。
    
    \end{itemize}
\subsubsection{词法分析}
\begin{itemize}
    \item [1)]
    根据书后产生式,书写正则表达式,并使用别名命名,方便进一步的正则表达式的识别与匹配。\par

    \item [2)]
    完成正则表达式匹配后的动作,即需要建立叶子节点。\par

    \item [3)]
    根据示例中的错误进一步完善词法分析的功能,识别出更多的词法错误。\par

\end{itemize}
\subsubsection{语法分析}
\begin{itemize}
    \item [1)]
    首先在定义部分定义终结符与非终结符,并设置左右优先级。\par

    \item [2)]
    在规则段根据书后相关产生式书写相关规则,并根据规则中使用的符号构建语法树。\par
    若产生式为
    ExtDefList -> ExtDef ExtDefList\par
    则需要创建ExtDefList的节点,并将\$1插入到ExtDefList的子节点中,接下来将\$2插入\$1的兄弟节点。
    \item [3)]
    在第三段中,声明了错误恢复的方法,由于yacc的错误恢复的特性,实现的方法可以指出语法错误的位置及原因。\par
\end{itemize}


\subsection{优点}
\subsubsection{语法树的构建}
通过先构建父节点再构建叶节点的方式,使整体脉络更加清晰。如代码所示:\par
\lstset{language=C}
\begin{lstlisting}
    p = CreateNode(@$.first_line, NON_TOKEN, "ExtDefList", NULL);  
    InsertNode(p, $1); 
    InsertNode(p, $2); 
    $$ = p;
\end{lstlisting} 
\subsubsection{打印语法分析树}
引入枚举类型,为每一个节点分配类型,最终打印语法树时候,可以判断他的类型,只有在int、float、type、id时候才会打印。
\subsubsection{增强词法分析}
根据书后示例,进一步完善了词法分析过程,使程序能够识别十六进制及八进制的词法错误。



\section{如何编译程序}



实验的根目录下有makefile文件。\par
实验的内容的目录结构如下:
\dirtree{%
    .1 {Experiment1}.
    .2 {Code}.
    .3 {lexical.l}.
    .3 {syntax.y}.
    .3 {main.c}.
    .3 {makefile}.
    .3 {node.c}.
    .3 {node.h}.
    .2 {TestCase}.
    .3 {test1.cmm}.
    .3 {test2.cmm}.
    .3 {test3.cmm}.
    .2 {makefile}.
}
其中在Experiment根目录下的./makefile会与./Code/makefile嵌套调用。\par
本次实验中,所有指令在根目录执行即可,指令如下:\par
\begin{itemize}
    \item [1)] 
    编译./Code下的所有代码,并生成可执行文件./Code/parser.
    \lstset{language=bash}
    \begin{lstlisting}
        $make compile
    \end{lstlisting} 
    \item [2)] 
    清除./Code下所有编译产物。
    \lstset{language=bash}
    \begin{lstlisting}
        $make clean
    \end{lstlisting} 
    \item [3)] 
    测试示例。
    \lstset{language=bash}
    \begin{lstlisting}
        $make nessaryTest
    \end{lstlisting} 
\end{itemize}
位于实验目录的根目录执行编译和测试命令,即可完成测试。

\end{document}